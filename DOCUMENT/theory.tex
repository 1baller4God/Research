\documentclass[12pt, letterpaper, twoside]{article}
\usepackage[utf8]{inputenc}
\usepackage{amsmath}
\usepackage{amssymb}
\usepackage{graphicx}
\usepackage{epstopdf}
\usepackage{inputenc}
\usepackage{geometry}
\usepackage{booktabs}
\geometry{left=2.5cm,right=2.5cm,top=2.5cm,bottom=2.5cm}


\title{%
  Understanding Graph Theory \\
  
  \large In Application with Graph Coloring}
\author{ --insert name here-- \thanks{funded by the Kings University}}
\date{ May - August 2022 }

\begin{document}

\begin{titlepage}
\maketitle
\end{titlepage}

\begin{abstract}
Understanding Graph Theory. What is Graph Theory? 
Graph theory studies the mathematical structures used in moddelling the relationships between various elements.
The various elements, or nodes, are connected using edges. 
Nodes can be combined freely or with restrictions. In our study, we planted our attention on diverse restricted graph structures.
In line with graph theory, the application of Graph colouring became a primary outlet for research. Graph colouring is an essential application for studying restricted graphs. 
The colours act as labels for distinguishing elements on a constrained graph.
\end{abstract}

\begin{tabular}{l c p{.6\textwidth}}
\toprule
Terms           &                  & Definitions \\
\midrule
Chromatic number &$:\Leftrightarrow$& The smallest number of colours needed to color a graph (for restricted graphs).\\
Connected Graph  &$:\Leftrightarrow$& A graph in which all nodes are connected with edges. \\
Restricted Graph  &$:\Leftrightarrow$& A graph in which connected adjacent nodes cannot have the same label/color.\\
isomorphic Graph  &$:\Leftrightarrow$& A graph can take on different forms while having the same number of vertices and edges.\\

\bottomrule
\end{tabular}


\clearpage


\section{Critical Graphs}
Critical Graphs. For a graph to be considered critical, if any vertex or node is removed, the chromatic number must decrease in value. If any exception arises, the graph is therefore not critical.


\end{document}