\documentclass[11pt]{article}
\usepackage[margin=1.0in]{geometry}


%\input{paper_macros.tex}

\usepackage{verbatim}
\usepackage{hyperref}
\usepackage{amsmath}
\usepackage{amsthm}
%\usepackage{ulem} %useful for strikethough, but will underline journal names!
\usepackage{amssymb}
\usepackage{amsfonts}
\usepackage{multicol}
\usepackage{subfigure}
\usepackage[numbers,sort&compress]{natbib}
\usepackage{booktabs}
\usepackage{titlesec}
\usepackage{longtable,tabu}

\usepackage{enumitem}
\setlist[enumerate]{label={\arabic*.}}
%\setlist[enumerate]{label={\upshape(\alph*)}}

\usepackage{color}
\newcommand{\red}[1]{\textcolor{red}{#1}}
\newcommand{\M}[2]{M^{#1}_{#2}}
\newcommand{\infcrit}[2]{G(#1,#2)}

\newcommand{\den}{\mbox{Den}}
\newcommand{\Hstar}{{\cal H}^{*}}
\newcommand{\calH}{{\cal H}}

\newcommand{\change}[1]{\textcolor{blue}{#1}}

\usepackage{tikz}
\usetikzlibrary{calc}

\usepackage{tkz-graph}


\newtheorem{theorem}{Theorem}[section]
\newtheorem{lemma}[theorem]{Lemma}
\newtheorem{result}[theorem]{Result}
\newtheorem{corollary}[theorem]{Corollary}
\newtheorem{fact}[theorem]{Fact}
\newtheorem{claim}[theorem]{Claim}

\theoremstyle{definition}
\newtheorem{definition}[theorem]{Definition}
\newtheorem{remark}[theorem]{Remark}
\newtheorem{conjecture}[theorem]{Conjecture}
\newtheorem{problem}{Open Problem}

%SQUISHLISTS
\newcommand{\squishlist}{
 \begin{list}{$\bullet$}
  { \setlength{\itemsep}{0pt}
     \setlength{\parsep}{3pt}
     \setlength{\topsep}{3pt}
     \setlength{\partopsep}{0pt}
     \setlength{\leftmargin}{2.5em}
     \setlength{\labelwidth}{1em}
     \setlength{\labelsep}{0.5em} } }

\newcommand{\squishlisttwo}{
 \begin{list}{$\triangleright$}
  { \setlength{\itemsep}{0pt}
     \setlength{\parsep}{0pt}
    \setlength{\topsep}{0pt}
    \setlength{\partopsep}{0pt}
    \setlength{\leftmargin}{2em}
    \setlength{\labelwidth}{1.5em}
    \setlength{\labelsep}{0.5em} } }

\newcommand{\squishend}{
  \end{list}  }
  
\newcommand{\forbid}{$(P_4+P_1,\ K_3+P_1,2P_2)$}

\newcommand{\noneighbs}{\overline{N[v]}}

\begin{document}

\title{Critical \forbid -free graphs}
\author{
Ben Cameron\\ %\thanks{Research partially supported by the Natural Sciences and Engineering Research Council of Canada (NSERC) (grants DGECR-2022-00446 and RGPIN-2022-03697)}\\
\small Department of Computing Science\\
\small The King's University\\
\small Edmonton, AB Canada\\
\small ben.cameron@kingsu.ca\\
\and
Thaler Knodel\\ %\thanks{Research partially supported by the Natural Sciences and Engineering Research Council of Canada (NSERC) (grants DGECR-2022-00446 and RGPIN-2022-03697)}\\
\small Department of Computing Science\\
\small The King's University\\
\small Edmonton, AB Canada\\
%\small ben.cameron@kingsu.ca\\
%Ch\'{i}nh T. Ho\`{a}ng\\ %\thanks{Research partially supported by the Natural Sciences and Engineering Research Council of Canada (NSERC) (grants DGECR-2022-00446 and RGPIN-2022-03697)}\\
%\small Department of Physics and Computer Science\\
%\small Wilfrid Laurier University\\
%\small Waterloo, ON Canada\\
%\small choang@wlu.ca\\
}

\date{\today}

\maketitle


%=================================================================
%=================================================================
\begin{abstract}
%A graph is $k$-vertex-critical if $\chi(G)=k$ but $\chi(G-v)<k$ for all $v\in V(G)$. 
%It is known that there are only finitely many $k$-vertex-critical $(P_5,\text{ gem})$-free graphs 
%We give a new, stronger proof that there are only finitely many  $k$-vertex-critical ($P_5$,~gem)-free graphs for all $k$. Our proof further refines the structure of these graphs and allows for the implementation of a simple exhaustive computer search to completely list all $6$- and $7$-vertex-critical $(P_5$, gem)-free graphs. Our results imply the existence of polynomial-time certifying algorithms to decide the $k$-colourability of $(P_5$, gem)-free graphs for all $k$ where the certificate is either a $k$-colouring or a $(k+1)$-vertex-critical induced subgraph. Our complete lists for $k\le 7$ allow for the implementation of these algorithms for all $k\le 6$.
\end{abstract}

\section{Structure}

Throughout this section, assume $G$ is a $k$-vertex-critical non-complete \forbid -free graph.\\

For any $v\in V(G)$, $G$ partitions into $\{v\}$, $N(v)$, and $\overline{N[v]}$. $N(v)$ further partitions into $N_1, N_1', N_2$. $\exists u_1 \in \noneighbs$ such that $u_1 \sim N_1$, and $\exists u_2 \in \noneighbs$ such that $u_2 \sim N_2$.


\begin{lemma}\label{lem:P4K3-free}
For every vertex $v\in V(G)$,  $G[\noneighbs]$ is $(P4, K_3)$-free.
\end{lemma}
\begin{proof}
If $S\subseteq\noneighbs$ induces $P_4$ or $K_3$, then $\{v\}\cup S$ induces a $P_4+P_1$ or $K_3+P_1$, a contradiction.
\end{proof}



\begin{lemma}\label{lem:completebipartite}
For every vertex $v\in V(G)$,  $\noneighbs$ induces $K_{n,m}$ some $n,m\ge 1$.
\end{lemma}
\begin{proof}
By Lemma~\ref{lem:P4K3-free}, $\noneighbs$ is $P_4$-free and therefore either a join of disjoint union of graphs (since $P_4$-free graphs are co-graphs). Thus, every componenet is $K_1$ or a join. Further, since $\noneighbs$ is $K_3$-free, each component must be a complete bipartite graph, since if it were the join of any graph with an edge, a triangle would be induced, contradicting Lemma~\ref{lem:P4K3-free}. Now, if some component of $\noneighbs$ is $K_1$, then the neighbourhood of this component is contained in the neighbourhood of $v$ which makes comparable vertices and therefore contradicts $G$ being vertex-critical. Thus, every component of $\noneighbs$ contains at least one edge. If there are two components, then take any two vertices that are adjacent from two different components and these four vertices will induce a $2P_2$, contradicting $G$ being $2P_2$-free. Thus,  $\noneighbs$ induces $K_{n,m}$ some $n,m\ge 1$.
\end{proof}

\begin{lemma}\label{lem:neighbscomplete}
$N_1 - N'_1$ is complete to $N_2$.
\end{lemma}
\begin{proof}
If $N_1 - N'_1$ is not complete to $N_2$, then $n_1 \in N_1, n_2 \in N_2$, $\{ n_1, u_2, n_2, u_1 \}$ induces a $2K_2$.
\end{proof}
\begin{lemma}\label{lem:n2indep}
$N_2$ is an independent set.
\end{lemma}
\begin{proof}
By ~\ref{lem:completebipartite}, the non-neighbours of $N_1$ are a complete bipartite graph.
\end{proof}

Now let $S$ be the maximum independent set of $G$, and $v \in S$. Let $S - v = \{u_1, u_2, \dots, u_l \}$. Also, let $V(G) - S - N(v) = \{Y_1, Y_2, \dots, Y_j\}$. Suppose that $l$ is greater than some arbitrary constant, and that $j \geq 2$. If $Y_1$ is complete to $N(v)$, then $Y_1, v$ are comparable, which contradicts the $k$-vertex criticality of the graph. So let $N' \subseteq N(v)$ such that $Y_1$ is anticomplete to $N'$.

\begin{lemma}\label{lem:anticomplete}
$Y_i$ is anticomplete to $N'$ $\forall i \in \{1, \dots, j\}$.
\end{lemma}
\begin{proof}
If $\exists Y_i \in V(G) - S - N(v)$ such that $Y_i \sim n$, then $\{ u_1, Y_i, n \} \subseteq \overline{N(Y_1)}$ contradicting that the non-neighbours are a complete biparite graph. 
\end{proof}

\begin{lemma}\label{lem:ScompleteNprime}
$S- v$ is complete to $N'$.
\end{lemma}
\begin{proof}
If $\exists v_i \in S - v$ such that $u_i \not \sim n$ for some $n \in N'$, then $\{ u_i, Y_1, v, n \}$ induces a $2K_2$, which contradicts our graph characterization.
\end{proof}

So $N(Y_i) \subseteq N(Y_1) \forall i \in \{2, \dots, j\}$, so $Y_i, Y_1$ are comparable vertices, making this not vertex critical and thus a contradiction. Further, since this applies for all $i \geq 2$, then the $|Y| \leq 1$. We also know (somehow) that this applies to the set of $\{u_1, \dots, u_l \}$ so they are limited to $\leq 1$.
Now we can use Ramsay's Theorem to identify that since there is a maximum independent set, there is a finite amount of graphs.

\section{$claw + P_1, K_3 + P_1, 2P_2$-free}
We can use a similar technique to find a finite amount of these graphs. Assume we have a $claw + P_1, K_3 + P_1, 2K_2$-free graph. Then we have the maximum independent set $S$. Let $v \in S$. We will reuse our definition of $\noneighbs$.
\begin{lemma}\label{lem:clawminset}
$\forall v \in \noneighbs, $
\end{lemma}

\section{$(2P_2, P_4 + P_1, chair, bull)$-free}
Let $G$ be a $(2P_2, P_4 + P_1, chair, bull)$-free $k$-vertex-critical graph. Let us have a vertex $v$ and $N(v)$ and $\noneighbs$. Let $S$ be the maximum independent set of $G$.
\begin{lemma}\label{lem:

Results\\
$P_4 + \ell_1 P_1, 2P_2, \ell_2 squid$\\s
$claw + P_1, K_3 + P_1, 2P_2$\\
$2P_2, P_4 + P_1, chair, raging bull$

%Let $S$ be a maximum stable set of $G$. Then, each vertex in $G-S$ has at least one neighbour in $S$. Let $A$ be the vertices in $G-S$ with exactly one neighbour in $S$. Let $B = G - (S \cup A)$. We note that every vertex in $B$ shas at least two neighbours in $S$. Let $S_A$ be the set of vertices $s \in S$ such that some vertex in $A$ is adjacent to $s$. Let $S_B = S - S_A$.
%
%
%\begin{lemma}\label{lem:sAs'Aanticomplete}
%For any two vertices $s,s' \in S_A$, $s_A$ is anti-complete to $s'_A$.
%\end{lemma}
%\begin{proof}
%Suppose $s,s'\in S_A$, $a_1\in s_A$ and $a_2\in s'_A$ such that $a_1\sim a_2$. Then the set $\{a_1,a_2,s,s',s''\}$ induces a $P_4+P_1$ for any $s''\in S-\{s,s'\}$, a contradiction.
%\end{proof}
%
%\begin{lemma}\label{lem:sAisaclique}
%DO WE NEED? If $|S_A|\ge 2$, then $s_A$ induces a clique for all $s \in S_A$.
%\end{lemma}
%\begin{proof}
%Suppose $|S_A|\ge 2$ and let $s,s'\in S_A$. Without loss of generality, suppose that $a_1,a_2\in s_A$ such that $a_1\nsim a_2$. Now, since $s\nsim s'$ by Lemma~\ref{lem:sAs'Aanticomplete}, $\{s,s',a_1,a_2,a'\}$ induces a $P_3+P_2$ for any $a'\in s'_A$, a contradiction.
%\end{proof}
%
%
%\begin{lemma}\label{lem:Bnonempty}
%$B\neq\emptyset$.
%\end{lemma}
%\begin{proof}
%Suppose $B=\emptyset$, so $S=S_A$. If $|S|<2$, then $G$ must be $K_k$ since it is $k$-vertex-critical, a contradiction. So, $|S|\ge 2$, and therefore by Lemma~\ref{lem:sAs'Aanticomplete}, $G$ is disconnected and therefore not critical, a contradiction.
%\end{proof}
%
%
%
%\section*{Acknowledgements}
%This work was supported by the Canadian Tri-Council Research Support Fund. Each author was supported by individual NSERC Discovery Grants.

\bibliographystyle{abbrv}
\bibliography{refs}


\end{document}

